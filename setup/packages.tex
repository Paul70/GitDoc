% ################################################
% This file includes all packages
% Do not inlcude additional packages elsewhere!
% ################################################

% ----------------------------------------------------------------------------------------------------------------------
% Basic encoding, languages and the scrhack package
%
% Due to compability problems between the KOMA package and some old packages (like listings, float etc.) 
% Mr. Kohm (the author of KOMA) wrote the scrhack package to overcome this issue. 
% For more information on this topic see chapter 16 of the KOMA documentation.
% It is recomanded to load the scrhack package at an early state and before other fancy 
% packages!
% ----------------------------------------------------------------------------------------------------------------------
\usepackage[T1]{fontenc}                        % use 256 glyph encoding (standard for western european languages)
\usepackage[ngerman,english]{babel}             % load language package; default language is the last language listed, 
												% the other given languages are optional 		
\usepackage{scrhack}							% needed for older packages to work smoothly with KOMA (e.g. listings)										


% ----------------------------------------------------------------------------------------------------------------------
% Page layout packages
% ----------------------------------------------------------------------------------------------------------------------	
\usepackage{scrlayer-scrpage}					% customize header, footer etc. This package replaces fancyhdr-package
												% and works in combination with the KOMA-script document classes
\usepackage{geometry}							% define new page geometries


% ----------------------------------------------------------------------------------------------------------------------
% Bibliography, glossary, acronyms packages
% ----------------------------------------------------------------------------------------------------------------------
\usepackage[style = numeric]{biblatex}			% for referencing and list of referenced, we use biber 
\usepackage{csquotes}							% loading this package is recommended if you use biblatex package
\usepackage{acro}								% enables automated abbrevations and acronyms listing


% ----------------------------------------------------------------------------------------------------------------------
% All other packages for the fancy stuff
% ----------------------------------------------------------------------------------------------------------------------
\usepackage{graphicx}                           % make use of color, graphics, etc...	
\usepackage{xcolor}                             % you can use and define RGB colors
\usepackage{soul}
\usepackage{scrhack}							% needed for the listings package
\usepackage{listings}							% text which looks like software code
\usepackage{lastpage}							 % includes internal page counter
%\usepackage{amssymb}                            % use all available math symbols
%\usepackage{trfsigns}	                         % use math transformation signs, e.g. the Laplace transformation sign
%\usepackage{mathtools}							 % for math environments equation, split, align, multiline etc.
%\usepackage{enumerate}                          % for enumeration, e.g. equations
\usepackage[figure,table]{totalcount}			% counter for figures and tables, accessible through \totalfigures and \totaltables commands
\usepackage{totcount}							% for counting references
\usepackage{float}								% for working with float values, e.g. \begin{figure}[H]
%\usepackage{wasysym}                            % for the use of checkboxes (\Box, \XBox)
\usepackage{tikz}								% for tikz
\usepackage{tikzscale} 							% for tikz
\usepackage{pgfplots}                           % for generating high quality plots (very nice!)
%\usepackage{longtable}							 % for tables
%\usepackage{threeparttable}					 % for tables
%\usepackage{multirow}							 % for tables
%\usepackage{tabularx}							 % for tables
%\usepackage{booktabs}							 % for tables
%\usepackage{colortbl}							 % for tables
%\usepackage{multicol}							 % for tables
%\usepackage[ruled,vlined,linesnumbered,algochapter]{algorithm2e}

			
% ----------------------------------------------------------------------------------------------------------------------
% Optional packages 
%
% These packages are not neccesary for the document's content but offer some helpful
% features.
% ----------------------------------------------------------------------------------------------------------------------								
\usepackage{blindtext}							% enables blindtext functioanlity
\usepackage{layout}								% enables \layout command (prints the page layout picture)


% ----------------------------------------------------------------------------------------------------------------------
% Hyperref packages 
% Add these two packages always at the end. 
% ----------------------------------------------------------------------------------------------------------------------
\usepackage{hyperref}							% creates links and refs inside your PDF document
\usepackage[capitalise]{cleveref}       		% must be loaded after hyperref