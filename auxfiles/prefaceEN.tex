The intention of this document is not to write an absolute beginners guide for git or explain everything in detail again. 
Therefore, I recommend reading other articles or references throughout the text which helped me a lot to understand how Git really
works under the surface. 
\\
This report is more a protocol of 
the problems I encountered when working with git so far. I summarize these problems in so called "User Stories"
(they are highlighted in yellow within the text) and describe my solutions to cope with this tasks. Alongside to these stories,
theory is given to understand the solutions presented.



The command line is the tool to work with Git in this document. I know that there exists a lot of graphical clients such as 
\textit{Tortoise Git} or \textit{GitKraken} \cite{Unknown2021Tortoise,Unknown2021Kraken}. However, I recommend using the 
command line only. Then, you have to know and to understand what you are doing. This is especially helpful if you are still 
learning to work with Git effectively.


As pointed out at the beginning, the focus is directed to practical problems. In case you want to learn a little bit more of 
Git's magic, I recommend reading \cite{2017wiegleygi}.
