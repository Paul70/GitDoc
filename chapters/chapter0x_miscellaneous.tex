\chapter{Miscellaneous}
\label{chapter:Miscellaneous}

\section{Setting VS Code as Default Git Editor}
\label{chapter:Miscellaneous.1}

VS Code is a very popular and handy editor. By editing git's global .config file it is possible to use VS Code as the default editor for writing 
new more verbose commits and to use it as git's merge and diff tool. The global .config file contains all important settings configuration parameter
like your e-mail address and your name which appears when you write a new commit, for example.
A display of this file is possible with
\begin{lstlisting}
	$\dollar$ git config --global -e
\end{lstlisting} 
The following steps are necessary to use VS Code as the default diff tool:
\begin{lstlisting}
	$\dollar$ git config --global diff.tool vscode
	$\dollar$ git config --global difftool.vscode.cmd 
	  'code --wait --diff $\dollar$LOCAL $\dollar$REMOTE'
\end{lstlisting}
For using VS Code as the default merge tool execute the next two commands:
\begin{lstlisting}
	$\dollar$ git config --global merge.tool vscode
	$\dollar$ git config --global difftool.vscode.cmd 'code --wait --diff $\dollar$MERGED'
\end{lstlisting}
Again, check your global .config file and see the changes made so for. Now your are able to use 
VS Code as your editor of choice. To also use it as the standard commit tool set it as the global editor in general by typing 
\begin{lstlisting}
	$\dollar$ git config --global core.editor "code --wait"
\end{lstlisting}
That's it! VS Code is now your overall git editor.


\section{Stashing}
\label{chapter:Miscellaneous.2}


\section{Submodules}
\label{chapter:Miscellaneous.3}



\section{The Benefits of having the Staging Area}
\label{chapter:Miscellaneous.4}

