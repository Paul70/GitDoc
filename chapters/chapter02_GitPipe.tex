\chapter{The (local) Git Forest;) - Working Tree, Staging Area and Commit Tree}
\label{chapter:2}

As shortly mentioned in \cref{chapter:1.3.1} the three major parts of your local git project are
\begin{itemize}
	\item the Working Tree or all the files and code you actually write,
	\item the Staging Area or Index Tree where you prepare new content and changes of your work ready to commit and 
	\item the Commit Tree resembling the whole history and development of your project. 
\end{itemize}
These three items make up the local Git forest which versions and organizes a project in an ongoing process. The meaning of the Working Tree and the 
Commit Tree is obvious but that holds not for the Staging Area. I will not spend time here explaining the benefits of the Index Tree 
since it is not important to understand what follows. The interested reader may refer to \cref{chapter:Miscellaneous.2} or 
\cite{sonulohaniWhat2021}. 
\\
This chapter is not meant to give another explanation of how to create new commits. The focus lies on these three trees 
and how you improve the maintenance and administration of the software development process with the help of Git. 

 


\section{Commit, HEAD and local Branches}
\label{chapter:2.1}

This section is focused on summarizing briefly Git's meaning of commits, branches 
and referencing or addressing commits, respectively. We concentrate our investigations on the local 
.git repository. Concepts of fetch and merge, i.e. a pull operation and remote branching is part of \cref{chapter:3}. 



\subsection*{Commits}
  
Every commit represents a loaf in your commit tree. The whole commit tree represents the 
progress of your project. A single loaf or commit encapsulates 
your whole project at the time when the commit was made. That is, a commit is a snapshot 
of your project and you can extract that project state from just this single commit.
Each commit is unique and accessible through its identifying SHA-1 hash key.

The notion of commits is fundamental when using git. \textbf{Git is all about commits!} Especially the high level end user API of 
git is all about creating commits, pointing to or referencing special commits, grouping commits together,
looking inside a past commit, cutting off commits from your tree, restoring them etc.
But always remember, one single commit represents and contains your whole project at the time the commit 
was made.
   
Next to the commits, Git offers a set of pointers which are directed to special commits by default. These 
pointers are for addressing commits and investigating your commit tree. The set consist of:
\begin{itemize}
	\item HEAD pointer
	\item branch pointers
	\item origin/master pointer (refers to your remote commit tree)
	\item master or main branch pointer
\end{itemize}

Sections xxx and yyy concentrate on how to work with these pointers using the checkout and reset 
commands.


\subsection*{Local Git Branches}

The term branch often implies a somehow heavyweight and complex construct as part of the commit 
tree. In fact the contrary is the case. In the context of Git, a branch is nothing more then 
a named pointer directed on the latest commit in an directed, acyclic (sub-)graph of development
in the commit tree. That's it! Again, a branch is only a named pointer to a commit and the 
branch name represents a colletion of all commits with the same ancestor in the tree. However, although
we speak of pointers, these branch pointers are not meant to be moved freely around in your commit tree.
They are more like references and always refer to the latest commit in such an acyclic graph.
In the following, branch means the whole acyclic (sub-)graph and the term branch pointer 
means the reference to the latest commit in this branch.
One special branch is the \textbf{master} or \textbf{main} branch.

\subsection*{Master/Main Branch}
The master or main branch represents the trunk of your commit tree and is crated 
with the git init command. The master or main branch pointer is always directed to the latest commit added to 
this master branch. That is, creating a new commit also moves the pointer one commit further.

\subsection*{HEAD pointer, detached HEAD State}
The HEAD pointer is also created with the git init command. At the beginning this pointer is attached to 
the master pointer and refers to the same commit. Attached means, that the HEAD pointer is moved together 
with the master pointer. However, HEAD is designed as a real pointer and meant to be moved around in 
your commit tree. It is the git tool to reference arbitrary commits.
If the HEAD pointer (and only the HEAD pointer) is decoupled from the master pointer or any other branch pointer and moved backwards in time, i.e. to an older 
commit in the tree, one calls this state a detached HEAD.
To sum it up, the HEAD is directed to the commit which represents the the project state as you can see it, when you 
look in your current working tree.

\subsection*{origin/master} 
This pointer refers to the remote master branch inside the remote .git repository. Since this section only 
covers the local commit tree, the reader may consult chapter xxx for more information on remote HEAD and master
as well as remote branching. 
\\
Finally, \cref{fig:GitAddCommit} shows a picture of the commit tree in relation with the other trees and references
presented. It depicts the situation while adding a new commit and shows how new data gets assembled.
The commit tree is highlighted in blue since we put new information up to this stage. The newly created commit C3 
is also highlighted since it contains this new information next to everything which is inside C2 and C1.
\begin{figure}[H]
	\centering
	\newcommand*{\xMin}{0}
\newcommand*{\xMax}{14}
\newcommand*{\yMin}{0}
\newcommand*{\yMax}{6}




\begin{tikzpicture}
%	 \foreach \i in {\xMin,...,\xMax} {
%		\draw [very thin,gray] (\i,\yMin) -- (\i,9)  node [below] at (\i,\yMin) {$\i$};
%	}
%	\foreach \i in {\yMin,...,9
%	} {
%		\draw [very thin,gray] (\xMin,\i) -- (\xMax,\i) node [left] at (\xMin,\i) {$\i$};
%	}

	\node[] at(1.8,8.2) {\textbf{{\footnotesize \textsf{Git Forest Layer}}}};
	\node[] at(1.8,0.4) {\textbf{{\footnotesize \textsf{Git Commit Tree Layer}}}};
	
	% input arrow
	\draw[-stealth,line width=3pt](1.0,6.7) -- (3.15,6.7);
	\node[] at(1.9,6.9) {\textbf{{\footnotesize \textsf{Input}}}};
	\node[] at(2.0,6.4) {\textbf{{\scriptsize \textsf{New Work}}}};
	
	% rectangle Working tree 
    \draw[line width=0.5mm,rounded corners=7pt](3.2,6) rectangle ++(2.0,1.4);
    \node[] at(4.18,6.85) {\textsf{Working}}; 
    \node[] at(4.18,6.45) {\textsf{Tree}};
    \draw[-stealth,line width=3pt](5.2,6.7) -- (6.55,6.7);
    \node[] at(5.7,6.9) {\textbf{{\footnotesize \textsf{add}}}};

	% rectangle Staging Area 
	\draw[line width=0.5mm,rounded corners=7pt](6.6,6) rectangle ++(2.0,1.4);
	\node[] at(7.55,6.85) {\textsf{Staging}}; 
	\node[] at(7.55,6.45) {\textsf{Area}};
	\draw[-stealth,line width=3pt](8.6,6.7) -- (10.15,6.7);
	\node[] at(9.25,6.9) {\textbf{{\footnotesize \textsf{commit}}}};
	
	% rectangle Commit tree 
	\draw[line width=0.5mm,rounded corners=7pt,fill=columbiablue](10.2,6) rectangle ++(2.0,1.4);
	\node[] at(11.15,6.85) {\textsf{Commit}}; 
	\node[] at(11.15,6.45) {\textsf{Tree}};
	
	% dashed line
	\draw[dashed,line width=2pt] (0.5,5.2) -- (13.5,5.2); 
	
	% commit tree
	\draw[line width=1.0pt] (1,3.5) circle (6mm);
	\node[] at(1,3.5) {\textsf{C1}};
	\draw[line width=2.0pt] (1.6,3.5) -- (2.4,3.5);
	\draw[line width=1.0pt] (3,3.5) circle (6mm);
	\node[] at(3,3.5) {\textsf{C2}};
	\draw[-stealth,line width=2pt](3,2.1) -- (3,2.86);
	\node[] at(3.0,1.9) {\textbf{{\footnotesize \textsf{master}}}};
	\node[] at(3.0,1.6) {\textbf{{\footnotesize \textsf{HEAD}}}};
	\node[] at(3.0,1.25) {\textbf{{\footnotesize \textsf{(old)}}}};
	\draw[line width=1.0pt,fill=pastelorange] (12.5,3.5) circle (6mm);
	\node[] at(12.5,3.5) {\textsf{C3}};
	\draw[-stealth,line width=2pt](12.5,2.1) -- (12.5,2.86);
	\node[] at(12.5,1.9) {\textbf{{\footnotesize \textsf{master}}}};
	\node[] at(12.5,1.6) {\textbf{{\footnotesize \textsf{HEAD}}}};
	\node[] at(12.5,1.25) {\textbf{{\footnotesize \textsf{(new)}}}};
	\draw[line width=2.0pt] (3.6,3.5) -- (11.9,3.5);
	\node[] at(7.5,3.75) {\textbf{{\footnotesize \textsf{new graph connection}}}}; 
	
	\draw[-{Implies},double, line width=1.5pt] (3,4.5) -- (3.4,5.8);
	\draw[{Implies}-,double, line width=1.5pt] (12.4,4.5) -- (12.0,5.8);
	
	%\draw [step=1.0,blue, very thick] (0.5,0.5) grid (1.5,1.5);
	%\draw [very thick, brown, step=1.0cm,xshift=-0.5cm, yshift=-0.5cm] (0.5,0.5) grid +(1.5,1.5);
	
\end{tikzpicture}
	\caption{Visualization of creating a new commit}
	\label{fig:GitAddCommit}
\end{figure}


After being familiar with the the Working Tree, the Staging Area, the Commit Tree, the pointer Git uses and 
the branch expression, you should read \cite{Unknown2021GitTools-Reset}. The source explains in depth the meaning 
of Git resetxxx.
What follows in \cref{chapter:2.2} and \cref{chapter:2.3} is a remainder of how to use git's reset and checkout commandsxxx.
 


\section{Git Reset}
\label{chapter:2.2}

Effectively using Git to organize your software projects requires at least a good understanding of the Git reset xxx 
command. After knowing how to create new commits with Git add xxx and Git commit, it is especially important to understand
how you can revert these actions or even reset your project to a past, stable state. In other words, Git reset xxx moves the
HEAD/master pointer to the specified commit.


The Git reset xxx command always refers to a commit specified by the user. Git offers several ways to address commits. The most
common one is using a commit's SHA-1 hash key. You do not have to type the whole key, the first five to six signs are enough:
\begin{lstlisting}
	$\dollar$ git reset <5-6 SHA1 digits> 
\end{lstlisting}
Another method often used is giving the position of a commit relatively to the HEAD pointer in the commit tree. For example, 
\begin{lstlisting}
	$\dollar$ git reset HEAD~2 
\end{lstlisting}
means commit C1 in \cref{fig:GitAddCommit}. Specifying no commit at all like
\begin{lstlisting}
	$\dollar$ git reset  
\end{lstlisting} 
is equivalent to 
\begin{lstlisting}
	$\dollar$ git reset HEAD~1 
\end{lstlisting}
which is the same as
\begin{lstlisting}
	$\dollar$ git reset HEAD~ 
\end{lstlisting}
Of course, these rules all apply for any other Git command requiring a commit as input argument. 


\subsection{Three Types of Git Reset}
\label{chapter:2.2.1}

There are three different ways of moving the HEAD/master pointer using Git reset xxx:
\begin{itemize}
	\item a soft reset,
	\item a mixed reset and
	\item a hard reset.
\end{itemize}


\subsubsection*{Git reset --soft}

A git reset --soft xxx is more or less the opposite command to git commit xxx. Resetting the Commit Tree softly
actually cuts off the latest commit, puts the information added with this commit back into the Staging Area and moves,
of course, the HEAD/master reference. Basically, \cref{fig:GitResetSoft} illustrates this situation  and 
the next use case summarizes this process. Again, the stage containing the latest modifications of the Working Tree is 
highlighted. Note, at this point, there exists no commit in the Commit Tree encapsulating theses changes.

\usrStory{I want to undo the last commit and put the files back into the staging area, that is perform a soft
reset.}

According to the different ways of addressing commits, there are the following, equivalent command options:
\begin{lstlisting}
	$\dollar$ git reset --soft <SHA1>  // or
	$\dollar$ git reset --soft HEAD~   // or
	$\dollar$ git reset --soft HEAD~1  // or
\end{lstlisting}
This is also possible for older commits by adjusting the SHA1 hash key or relative HEAD position. A soft reset with respect 
to commit C1 would put all uncommitted changes of C2 and C3 in the Staging Area or leave them untracked if the files were  
newly added in C2 or C3.
\begin{figure}[H]
	\centering
	\newcommand*{\xMin}{0}
\newcommand*{\xMax}{14}
\newcommand*{\yMin}{0}
\newcommand*{\yMax}{6}




\begin{tikzpicture}
%		 \foreach \i in {\xMin,...,\xMax} {
%				\draw [very thin,gray] (\i,\yMin) -- (\i,9)  node [below] at (\i,\yMin) {$\i$};
%			}
%		\foreach \i in {\yMin,...,9
%			} {
%				\draw [very thin,gray] (\xMin,\i) -- (\xMax,\i) node [left] at (\xMin,\i) {$\i$};
%			}
%	
	\node[] at(1.8,8.2) {\textbf{{\footnotesize \textsf{Git Forest Layer}}}};
	\node[] at(1.8,0.4) {\textbf{{\footnotesize \textsf{Git Commit Tree Layer}}}};
	
	
	% rectangle Working tree 
	\draw[line width=0.5mm,rounded corners=7pt](3.2,6) rectangle ++(2.0,1.4);
	\node[] at(4.18,6.85) {\textsf{Working}}; 
	\node[] at(4.18,6.45) {\textsf{Tree}};
	
	% rectangle Staging Area 
	\draw[line width=0.5mm,rounded corners=7pt,fill=columbiablue](6.6,6) rectangle ++(2.0,1.4);
	\node[] at(7.55,6.85) {\textsf{Staging}}; 
	\node[] at(7.55,6.45) {\textsf{Area}};
	\draw[stealth-,line width=3pt](8.6,6.7) -- (10.15,6.7);
	\node[] at(9.45,6.9) {\textbf{{\footnotesize \textsf{soft}}}};
	\node[] at(9.45,6.4) {\textbf{{\footnotesize \textsf{reset}}}};
	
	% rectangle Commit tree 
	\draw[line width=0.5mm,rounded corners=7pt](10.2,6) rectangle ++(2.0,1.4);
	\node[] at(11.15,6.85) {\textsf{Commit}}; 
	\node[] at(11.15,6.45) {\textsf{Tree}};
	
	% dashed line
	\draw[dashed,line width=2pt] (0.5,5.2) -- (13.5,5.2); 
	
	% commit tree
	\draw[line width=1.0pt] (1,3.5) circle (6mm);
	\node[] at(1,3.5) {\textsf{C1}};
	\draw[line width=2.0pt] (1.6,3.5) -- (2.4,3.5);
	\draw[line width=1.0pt] (3,3.5) circle (6mm);
	\node[] at(3,3.5) {\textsf{C2}};
	\draw[-stealth,line width=2pt](3,2.1) -- (3,2.86);
	\node[] at(3.0,1.9) {\textbf{{\footnotesize \textsf{master}}}};
	\node[] at(3.0,1.6) {\textbf{{\footnotesize \textsf{HEAD}}}};
	\node[] at(3.0,1.25) {\textbf{{\footnotesize \textsf{(new)}}}};
	\draw[line width=1.0pt] (12.5,3.5) circle (6mm);
	\node[] at(12.5,3.5) {\textsf{C3}};
	\draw[-stealth,line width=2pt](12.5,2.1) -- (12.5,2.86);
	\node[] at(12.5,1.9) {\textbf{{\footnotesize \textsf{master}}}};
	\node[] at(12.5,1.6) {\textbf{{\footnotesize \textsf{HEAD}}}};
	\node[] at(12.5,1.25) {\textbf{{\footnotesize \textsf{(old)}}}}; 
	
	\draw[stealth-,dashed,line width=1.0pt] (3.6,2.5) -- (11.9,2.5);
	\node[] at(7.5,2.75) {\textbf{{\footnotesize \textsf{move pointer}}}}; 
	\draw[-{Implies},double, line width=1.5pt] (12.4,4.5) -- (12.0,5.8);
	
\end{tikzpicture}
	\caption{Visualization of Git reset --soft}
	\label{fig:GitResetSoft}
\end{figure}
Note, Git reset does \textbf{not} delete any commits. In fact, Git actually never simply deletes any commits. That 
is not how Git works. Git reset xxx just cuts the connection in the commit tree. As  \cref{fig:GitResetSoft} implies, 
the commit C3 still exists and is accessible through Git's \textbf{reflog}. 
Handling the reflog is the topic of xxxyyy. At this point, we want to state, that Git never simply deletes commits and
it is fairly save to try things out.




\subsubsection*{Git reset --mixed}

Git reset mixed xxx is the second reset type. It is the same as Git reset soft plus it also un-stages the modifications.
The mixed reset type is also the default version of Git reset, \ac{IE} 
\begin{lstlisting}
	$\dollar$ git reset --mixed <SHA1>  
\end{lstlisting}
and
\begin{lstlisting}
	$\dollar$ git reset <SHA1>
\end{lstlisting}
is the same. To sum it up, Git reset mixed moves the HEAD/master pointer to the specified commit and leaves all modifications
un-staged in the Working Tree. It is the opposite command to Git add xxx. \Cref{fig:GitResetMixed} shows the situation after a mixed 
reset from commit C3 to C2. Consider the following use case:
\\
\usrStory{I want to completely reset my project. That is switching the HEAD/master 
pointer one (or more) commit(s) back and recreating the Working Tree stored in this commit. 
I DO want to keep the changes made so for inside the Working Tree}


The command for this operation is the one right above.

However, the behavior is slightly different, if there are currently files staged in the index. Then, a mixed reset does not move the HEAD/master
pointer and cut off the latest commit but simply un-stages all files, \ac{IE} only undoes Git add xxx.
\\
\usrStory{I want to un-stage all the files I recently put into the staging area via git add.}


To achieve this, just type 
\begin{lstlisting}
	$\dollar$ git reset 
\end{lstlisting}  
\begin{figure}[H]
	\centering
	\newcommand*{\xMin}{0}
\newcommand*{\xMax}{14}
\newcommand*{\yMin}{0}
\newcommand*{\yMax}{6}




\begin{tikzpicture}
	%		 \foreach \i in {\xMin,...,\xMax} {
		%				\draw [very thin,gray] (\i,\yMin) -- (\i,9)  node [below] at (\i,\yMin) {$\i$};
		%			}
	%		\foreach \i in {\yMin,...,9
		%			} {
		%				\draw [very thin,gray] (\xMin,\i) -- (\xMax,\i) node [left] at (\xMin,\i) {$\i$};
		%			}
	%	
	%	\node[] at(1.8,8.2) {\textbf{{\footnotesize \textsf{Git Forest Layer}}}};
	%	\node[] at(1.8,0.4) {\textbf{{\footnotesize \textsf{Git Commit Tree Layer}}}};
	
	\node[] at(1.8,8.2) {\textbf{{\footnotesize \textsf{Git Forest Layer}}}};
	\node[] at(1.8,0.4) {\textbf{{\footnotesize \textsf{Git Commit Tree Layer}}}};
	
	
	% rectangle Working tree 
	\draw[line width=0.5mm,rounded corners=7pt,fill=columbiablue](3.1,6) rectangle ++(2.0,1.4);
	\node[] at(4.18,6.85) {\textsf{Working}}; 
	\node[] at(4.18,6.45) {\textsf{Tree}};
	\draw[stealth-,line width=3pt](5.1,6.7) -- (6.6,6.7);
	\node[] at(6.0,6.9) {\textbf{{\footnotesize \textsf{mixed}}}};
	\node[] at(5.85,6.4) {\textbf{{\footnotesize \textsf{reset step}}}};
	
	% rectangle Staging Area 
	\draw[line width=0.5mm,rounded corners=7pt](6.6,6) rectangle ++(2.0,1.4);
	\node[] at(7.55,6.85) {\textsf{Staging}}; 
	\node[] at(7.55,6.45) {\textsf{Area}};
	\draw[stealth-,line width=3pt](8.6,6.7) -- (10.15,6.7);
	\node[] at(9.45,6.9) {\textbf{{\footnotesize \textsf{soft}}}};
	\node[] at(9.45,6.4) {\textbf{{\footnotesize \textsf{reset step}}}};
	
	% rectangle Commit tree 
	\draw[line width=0.5mm,rounded corners=7pt](10.2,6) rectangle ++(2.0,1.4);
	\node[] at(11.15,6.85) {\textsf{Commit}}; 
	\node[] at(11.15,6.45) {\textsf{Tree}};
	
	% dashed line
	\draw[dashed,line width=2pt] (0.5,5.2) -- (13.5,5.2); 
	
	% commit tree
	\draw[line width=1.0pt] (1,3.5) circle (6mm);
	\node[] at(1,3.5) {\textsf{C1}};
	\draw[line width=2.0pt] (1.6,3.5) -- (2.4,3.5);
	\draw[line width=1.0pt] (3,3.5) circle (6mm);
	\node[] at(3,3.5) {\textsf{C2}};
	\draw[-stealth,line width=2pt](3,2.1) -- (3,2.86);
	\node[] at(3.0,1.9) {\textbf{{\footnotesize \textsf{master}}}};
	\node[] at(3.0,1.6) {\textbf{{\footnotesize \textsf{HEAD}}}};
	\node[] at(3.0,1.25) {\textbf{{\footnotesize \textsf{(new)}}}};
	\draw[line width=1.0pt] (12.5,3.5) circle (6mm);
	\node[] at(12.5,3.5) {\textsf{C3}};
	\draw[-stealth,line width=2pt](12.5,2.1) -- (12.5,2.86);
	\node[] at(12.5,1.9) {\textbf{{\footnotesize \textsf{master}}}};
	\node[] at(12.5,1.6) {\textbf{{\footnotesize \textsf{HEAD}}}};
	\node[] at(12.5,1.25) {\textbf{{\footnotesize \textsf{(old)}}}}; 
	
	\draw[stealth-,dashed,line width=1.0pt] (3.6,2.5) -- (11.9,2.5);
	\node[] at(7.5,2.75) {\textbf{{\footnotesize \textsf{move pointer}}}}; 
	\draw[-{Implies},double, line width=1.5pt] (12.4,4.5) -- (12.0,5.8);
	
\end{tikzpicture}
	\caption{Visualization of Git reset --mixed}
	\label{fig:GitResetMixed}
\end{figure}




\subsubsection*{Git reset --hard}

The last reset type is Git reset hard xxx. Resetting the project hard means a soft reset plus a mixed reset plus removing all
modifications even from the Working Tree. A hard reset recreates your project as it is saved in the specified commit. After a hard reset
your Working Tree is clean, that is all modifications get removed. Please note, that this also implies, that all modifications which have 
not been saved in a commit are deleted. Therefore, before executing a hard reset, \textbf{ALWAYS} commit all your modifications made so far!
\cref{fig:GitResetHard} shows this situation.

\usrStory{I want to completely reset all the changes inside my Working Tree. That is switching the HEAD/master 
	pointer one (or more) commit(s) back and recreating the Working Tree stored in this commit. I do NOT want to keep 
	the changes made so for inside the Working Tree}

\begin{lstlisting}
	$\dollar$ git reset --hard <SHA1> 
\end{lstlisting} 

\begin{figure}[H]
	\centering
	\newcommand*{\xMin}{0}
\newcommand*{\xMax}{14}
\newcommand*{\yMin}{0}
\newcommand*{\yMax}{6}




\begin{tikzpicture}
	%		 \foreach \i in {\xMin,...,\xMax} {
		%				\draw [very thin,gray] (\i,\yMin) -- (\i,9)  node [below] at (\i,\yMin) {$\i$};
		%			}
	%		\foreach \i in {\yMin,...,9
		%			} {
		%				\draw [very thin,gray] (\xMin,\i) -- (\xMax,\i) node [left] at (\xMin,\i) {$\i$};
		%			}
	%	
	\node[] at(1.8,8.2) {\textbf{{\footnotesize \textsf{Git Forest Layer}}}};
	\node[] at(1.8,0.4) {\textbf{{\footnotesize \textsf{Git Commit Tree Layer}}}};
	
	
	% rectangle Working tree 
	\draw[line width=0.5mm,rounded corners=7pt](3.1,6) rectangle ++(2.0,1.4);
	\node[] at(4.18,6.85) {\textsf{Working}}; 
	\node[] at(4.18,6.45) {\textsf{Tree}};
	\draw[stealth-,line width=3pt](5.1,6.7) -- (6.6,6.7);
	\node[] at(6.0,6.9) {\textbf{{\footnotesize \textsf{mixed}}}};
	\node[] at(5.85,6.4) {\textbf{{\footnotesize \textsf{reset step}}}};
	\draw[stealth-,line width=3pt](1.0,6.7) -- (3.15,6.7);
	\node[] at(2.1,6.9) {\textbf{{\footnotesize \textsf{remove}}}};
	\node[] at(2.1,6.4) {\textbf{{\scriptsize \textsf{New Work}}}};
	
	% rectangle Staging Area 
	\draw[line width=0.5mm,rounded corners=7pt](6.6,6) rectangle ++(2.0,1.4);
	\node[] at(7.55,6.85) {\textsf{Staging}}; 
	\node[] at(7.55,6.45) {\textsf{Area}};
	\draw[stealth-,line width=3pt](8.6,6.7) -- (10.15,6.7);
	\node[] at(9.45,6.9) {\textbf{{\footnotesize \textsf{soft}}}};
	\node[] at(9.45,6.4) {\textbf{{\footnotesize \textsf{reset step}}}};
	
	% rectangle Commit tree 
	\draw[line width=0.5mm,rounded corners=7pt](10.2,6) rectangle ++(2.0,1.4);
	\node[] at(11.15,6.85) {\textsf{Commit}}; 
	\node[] at(11.15,6.45) {\textsf{Tree}};
	
	% dashed line
	\draw[dashed,line width=2pt] (0.5,5.2) -- (13.5,5.2); 
	
	% commit tree
	\draw[line width=1.0pt] (1,3.5) circle (6mm);
	\node[] at(1,3.5) {\textsf{C1}};
	\draw[line width=2.0pt] (1.6,3.5) -- (2.4,3.5);
	\draw[line width=1.0pt,fill=pastelorange] (3,3.5) circle (6mm);
	\node[] at(3,3.5) {\textsf{C2}};
	\draw[-stealth,line width=2pt](3,2.1) -- (3,2.86);
	\node[] at(3.0,1.9) {\textbf{{\footnotesize \textsf{master}}}};
	\node[] at(3.0,1.6) {\textbf{{\footnotesize \textsf{HEAD}}}};
	\node[] at(3.0,1.25) {\textbf{{\footnotesize \textsf{(new)}}}};
	\draw[line width=1.0pt] (12.5,3.5) circle (6mm);
	\node[] at(12.5,3.5) {\textsf{C3}};
	\draw[-stealth,line width=2pt](12.5,2.1) -- (12.5,2.86);
	\node[] at(12.5,1.9) {\textbf{{\footnotesize \textsf{master}}}};
	\node[] at(12.5,1.6) {\textbf{{\footnotesize \textsf{HEAD}}}};
	\node[] at(12.5,1.25) {\textbf{{\footnotesize \textsf{(old)}}}}; 
	
	\draw[stealth-,dashed,line width=1.0pt] (3.6,2.5) -- (11.9,2.5);
	\node[] at(7.5,2.75) {\textbf{{\footnotesize \textsf{move pointer}}}}; 
	\draw[-{Implies},double, line width=1.5pt] (12.4,4.5) -- (12.0,5.8);
	
\end{tikzpicture}
	\caption{Visualization of Git reset --hard}
	\label{fig:GitResetHard}
\end{figure}

 


\subsection{Reset with Path Specification}
\label{chapter:2.2.2}

\section{Git Checkout, Switch and Restore}
\label{chapter:2.3}

\subsection{Detaching the HEAD}
\label{chapter:2.3.1}

\subsection{Git Restore}
\label{chapter:2.3.2}

Hello World!

%Finally, an example of how to add literature into the written text \cite{meyers2005effective}.

%It is also possible to specify pages with \cite[see][page 127]{arens2015mathematik}.