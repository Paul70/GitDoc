\chapter{All about Branches}
\label{chapter:3}

\cref{chapter:2.1} shortly introduced the notion of branches and mentioned they are more or less special, named commits.
We have now a closer look at them and show the basic Git branch handling.

Git basically offers four types of branches, \textbf{local non-tracking branches}, \textbf{local tracking branches}, \textbf{remote tracking 
branches} and real \textbf{remote branches}.
The first three types only deal with your local git repository as the name implies. Only the last one really lives 
outside of your local system in a remote Git repository far, far away.
To list all branches currently available, whether local or remote, type 

\begin{lstlisting}
	$\dollar$ git branch -a 
\end{lstlisting}

The local git repository may be connected to several other remote system, see \cref{chapter:1.3.1}. You can view the list of 
all remote locations available via
\begin{lstlisting}
	$\dollar$ git remote -v 
\end{lstlisting}
 
In other words, these are the locations where you can pull from or push to. However, pulling or pushing from remote 
requires a little bit more then only the existence and knowledge from at least one remote git repository. Git xxx also needs 
to know if there are associated branches on the local and remote system, e.g. the local and remote master branch are connected 
with each other (see \cref{chapter:1.3.1} till \cref{chapter:1.3.3}). Pulling and pushing only works if git knows the remote 
counter parts of your local branches. If there are no remote branches specified for one local branch, pulling and pushing is not possible.



\section{Local Branches}
\label{chapter:3.1}


A local branch is a branch that only you (the local user) can see. It exists only on your local machine.
You can get a list of all local branches with
\begin{lstlisting}
	$\dollar$ git branch 
\end{lstlisting}



\subsection*{Local Non-Tracking Branches}

All local branches which are not associated with any other branch are called \textbf{local non-tracking branches}. They are not meant to 
be published on some remote server and are not designed to be used collaboratively. You use them for private 
development tasks on your local machine. It is not possible to do push or pull actions on those branches.

\usrStory{How do I crate a local non-tracking branch on my machine?}

There are several possibilities using Git's branch, checkout or switch command:
\begin{lstlisting}
	$\dollar$ git branch <branch_name>
	$\dollar$ git checkout -b <branch_name>  # also switches to the new branch
	$\dollar$ git switch -c <branch_name>    # also switches to the new branch 
\end{lstlisting}



\subsection*{Local Tracking Branches}

These branches are associated with another branch, i.e. they keep track of another branch, usually they track 
their remote counterpart. 
\begin{lstlisting}
	$\dollar$ git branch -vv
\end{lstlisting}
shows if a local branch has an associated remote branch. For example the master branch is a local tracking  
branch since it is mostly associated with it's remote version referred to as origin/master (or origin/main).
\\
Locale tracking branches are meant to work collaboratively on them. 

\usrStory{How can I start a local tracking branch based on a real remote branch?}

\begin{lstlisting}
	$\dollar$ git switch origin/<remote_tracking_branch_name>
\end{lstlisting}

They \verb|origin| keyword is the name of the remote repository. If your remote repository has another name, you have to replace
\verb|origin| with that name in the above command.  

Now, pull and push operation are possible on this branch.



\subsection*{Remote Tracking Branch}


All remote tracking branches available are listed with
\begin{lstlisting}
	$\dollar$ git branch -r
\end{lstlisting}

A local remote tracking branch gets automatically added to your local git repository when adding a 
local tracking branch. Think of your local remote-tracking branches as your local cache for what the real remote branches contain.
It is an local image of the remote branch which get updated through git fetch and since 
git fetch is part of git pull you update this image which each pull request. Actually you do not work directly with this branch as long as you just use 
pull and push.
\\
Read the section xxx about fetch and merge instead of pull to understand the purpose of this institution.
Even though all the data for a remote-tracking branch is stored locally on your machine (like a cache), it's still never called a local branch. 
(At least, I wouldn't call it that!) It's just called a remote-tracking branch.



\section{Remote Branches}
\label{chapter:3.2}

A remote branch is a branch on a remote location (in most cases origin). 
You can view all the remote branches (that is, the branches on the remote machine), by running 

\begin{lstlisting}
	$\dollar$ git remote show origin/or_any_other_remote_location_available
\end{lstlisting}

It is not necessary to have a local counterpart of each remote branch on your local system. You can create any time
a local remote tracking branch associated with an existing remote branch, see \cref{chapter:3.1} - local tracking branches.

\section{Overview}
\label{chapter:3.3}

\cref{fig:GitBranchTypes} shows the relations between these types of branches graphically.
\begin{figure}[H]
	\centering
	\newcommand*{\xMin}{0}
\newcommand*{\xMax}{14}
\newcommand*{\yMin}{0}
\newcommand*{\yMax}{6}




\begin{tikzpicture}
%			 \foreach \i in {\xMin,...,\xMax} {
%						\draw [very thin,gray] (\i,\yMin) -- (\i,9)  node [below] at (\i,\yMin) {$\i$};
%					}
%			\foreach \i in {\yMin,...,9
%					} {
%						\draw [very thin,gray] (\xMin,\i) -- (\xMax,\i) node [left] at (\xMin,\i) {$\i$};
%					}
		
	\node[] at(9.5,8.2) {\textbf{{\footnotesize \textsf{local System}}}};
	\node[] at(12.8,8.2) {\textbf{{\footnotesize \textsf{remote System}}}};
	
	
	% rectangle abstract branch
	\draw[line width=0.5mm,rounded corners=7pt](6,7.5) rectangle ++(2.0,1.4);
	\node[] at(7,8.4) {\textsf{Branch}};
	\node[] at(7,8.0) {\textsf{Types}};
	
	% rectangle local non tracking branch
	\draw[line width=0.5mm,rounded corners=7pt](0.0,5.0) rectangle ++(2.1,1.4);
	\node[] at(1.0,6.1) {\textsf{local}};
	\node[] at(1.05,5.7) {\textsf{non tracking}};
	\node[] at(1.0,5.3) {\textsf{Branch}};
	
	% rectangle local tracking branch
	\draw[line width=0.5mm,rounded corners=7pt](4.0,5.0) rectangle ++(2.0,1.4);
	\node[] at(5.0,6.1) {\textsf{local}};
	\node[] at(5.0,5.7) {\textsf{tracking}};
	\node[] at(5.0,5.3) {\textsf{Branch}};
	
	% rectangle remote tracking branch
	\draw[line width=0.5mm,rounded corners=7pt](8.0,5.0) rectangle ++(2.0,1.4);
	\node[] at(9.0,6.1) {\textsf{remote}};
	\node[] at(9.0,5.7) {\textsf{tracking}};
	\node[] at(9.0,5.3) {\textsf{Branch}}; 
	
	% rectangle remote branch
	\draw[line width=0.5mm,rounded corners=7pt](12,5.0) rectangle ++(2.0,1.4);	
	\node[] at(13.0,5.9) {\textsf{remote}};
	\node[] at(13.0,5.5) {\textsf{Branch}};
	
	% connection lines
	\draw[line width=1.0pt] (7.0,7.5) -- (7.0,7.0);
	\draw[line width=1.0pt] (1.0,7.0) -- (1.0,6.4);
	\draw[line width=1.0pt] (5.0,7.0) -- (5.0,6.4);
	\draw[line width=1.0pt] (9.0,7.0) -- (9.0,6.4);
	\draw[line width=1.0pt] (13.0,7.0) -- (13.0,6.4);
	\draw[line width=1.0pt] (1.0,7.0) -- (13.0,7.0);
	
	% arrows
	\draw[-stealth,line width=2pt](2.1,5.6) -- (4.0,5.6);
	\node[] at(3.05,5.8) {{\scriptsize\textsf{set-upstream}}};
	
	\draw[stealth-,line width=2pt](6.0,5.6) -- (8.0,5.6);
	\node[] at(7.0,5.8) {{\scriptsize\textsf{merge}}};
	
	\draw[stealth-,line width=2pt](10.0,5.6) -- (12.0,5.6);
	\node[] at(11.0,5.8) {{\scriptsize\textsf{fetch}}};
	
	\draw[stealth-,line width=2pt](5.0,5.0) -- (5.0,4.0);
	\draw[stealth-,line width=2pt](13.0,5.0) -- (13.0,4.0);
	\draw[line width=2pt] (5.0,4.0) -- (13.0,4.0);
	\node[] at(9.0,4.2) {{\scriptsize\textsf{push/pull}}};
	
	% dashed line
	\draw[dashed,line width=2pt] (11.0,5.3) -- (11.0,3.0);
	\draw[dashed,line width=2pt] (11.0,9.0) -- (11.0,6.3); 
	
	% commit tree
%	\draw[line width=1.0pt] (1,3.5) circle (6mm);
%	\node[] at(1,3.5) {\textsf{C1}};
%	\draw[line width=2.0pt] (1.6,3.5) -- (2.4,3.5);
%	\draw[line width=1.0pt,fill=pastelorange] (3,3.5) circle (6mm);
%	\node[] at(3,3.5) {\textsf{C2}};
%	\draw[-stealth,line width=2pt](3,2.1) -- (3,2.86);
%	\node[] at(3.0,1.9) {\textbf{{\footnotesize \textsf{master}}}};
%	\node[] at(3.0,1.6) {\textbf{{\footnotesize \textsf{HEAD}}}};
%	\node[] at(3.0,1.25) {\textbf{{\footnotesize \textsf{(new)}}}};
%	\draw[line width=1.0pt] (12.5,3.5) circle (6mm);
%	\node[] at(12.5,3.5) {\textsf{C3}};
%	\draw[-stealth,line width=2pt](12.5,2.1) -- (12.5,2.86);
%	\node[] at(12.5,1.9) {\textbf{{\footnotesize \textsf{master}}}};
%	\node[] at(12.5,1.6) {\textbf{{\footnotesize \textsf{HEAD}}}};
%	\node[] at(12.5,1.25) {\textbf{{\footnotesize \textsf{(old)}}}}; 
%	
%	\draw[stealth-,dashed,line width=1.0pt] (3.6,2.5) -- (11.9,2.5);
%	\node[] at(7.5,2.75) {\textbf{{\footnotesize \textsf{move pointer}}}}; 
%	\draw[-{Implies},double, line width=1.5pt] (12.4,4.5) -- (12.0,5.8);
	
\end{tikzpicture}
	\caption{Branch Types and their Relations}
	\label{fig:GitBranchTypes}
\end{figure}



\section{Advanced Dealing with Branches}
\label{chapter:3.4}



\subsection{Transforming a Local Non-Tracking Branch into a Local Tracking Branch}
\label{chapter:3.4.1}

\usrStory{I have created a local (non-tracking) branch and now I want to 
	publish this branch on the remote repository so that others can pull from and push to that 
	branch.}

In other words, you want to transform a local non-tracking branch into a local tracking branch by creating a real 
remote branch version of your simply local branch. To achieve this, you just 
have to set an upstream location for your local branch exactly as you did it with your master branch when setting 
up a remote location, see \cref{chapter:1.3} (note, \verb|-u| is short for \verb|--set-upstream|):

\begin{lstlisting}
	$\dollar$ git remote -u origin <local_branch_name>
\end{lstlisting}

It is common practice to name the remote branch after your local branch.
\\
This operation also adds a remote tracking branch locally to your Git repository. Local tracking branches and remote
tracking branches always come together.



\subsection{Merging Branches}
\label{chapter:3.4.2}

The very basic meaning of a Git merge is to create a new commit which has two ancestors, the so called merge 
commit:
\begin{figure}[H]
	\centering
	\newcommand*{\xMin}{0}
\newcommand*{\xMax}{14}
\newcommand*{\yMin}{0}
\newcommand*{\yMax}{9}

\begin{tikzpicture}
	% grid
%	\foreach \i in {\xMin,...,\xMax} {
%		\draw [very thin,gray] (\i,\yMin) -- (\i,\yMax)  node [below] at (\i,\yMin) {$\i$};
%	}
%	\foreach \i in {\yMin,...,\yMax} {
%		\draw [very thin,gray] (\xMin,\i) -- (\xMax,\i) node [left] at (\xMin,\i) {$\i$};
%	}
	
	% master branch
	\draw[line width=1.0pt] (1,7.5) circle (6mm);
	\node[] at(1,7.5) {\textsf{C1}};
	\draw[line width=2.0pt] (1.6,7.5) -- (2.4,7.5);
	\draw[line width=1.0pt] (3,7.5) circle (6mm);
	\node[] at(3,7.5) {\textsf{C2}};
	\draw[line width=1.0pt] (5.0,7.5) circle (6mm);
	\draw[line width=2.0pt] (3.6,7.5) -- (4.4,7.5);
	\node[] at(5.0,7.5) {\textsf{C3}};
	
	% local branch
	\draw[line width=1.0pt] (5.0,5.5) circle (6mm);
	\node[] at(5.0,5.5) {\textsf{B1}};
	\draw[line width=2.0pt] (5.6,5.5) -- (6.4,5.5);
	\draw[line width=2.0pt] (3.0,6.9) -- (4.4,5.5);
	\draw[line width=1.0pt] (7,5.5) circle (6mm);
	\node[] at(7,5.5) {\textsf{B2}};
	\draw[line width=1.0pt] (9.0,5.5) circle (6mm);
	\draw[line width=2.0pt] (7.6,5.5) -- (8.4,5.5);
	\node[] at(9.0,5.5) {\textsf{B3}};

	% merge commit
	\draw[line width=1.0pt,fill=pastelorange] (11.0,7.5) circle (6mm);
	\node[] at(11.0,7.6) {{\scriptsize \textsf{merge}}};
	\node[] at(11.0,7.4) {{\scriptsize \textsf{Commit}}};
	\draw[line width=2.0pt] (5.6,7.5) -- (10.4,7.5);
	\draw[line width=2.0pt] (9.6,5.5) -- (11.0,6.9);
	
	% head master pointer
	\draw[-stealth,line width=2pt](11.0,5.8) -- (11.0,6.7);
	\node[] at(11.0,5.6) {\textbf{{\footnotesize \textsf{master}}}};
	\node[] at(11.0,5.3) {\textbf{{\footnotesize \textsf{HEAD}}}}; 
	
	
\end{tikzpicture}
	\caption{Merge Commit}
	\label{fig:mergeCommit}
\end{figure}
This orange merge commit gets created when executing Git's \verb|merge| command.



\subsubsection*{Merging a local non-tracking Branch into master}

\usrStory{I have finished work on a local non-tracking branch, local\_branch, and now I want 
	to merge local\_branch into master branch.}

First of all, clean up your stages by committing every modifications on both branches. Do 
not continue unless everything is clean.

Merging is always done on the branch into which you 
want to merge the other branch. That is, for merging \verb|local_branch| into master, you have to switch 
to the master branch. And there is another thing. Conflicts! If you are sure, that 
there will be no conflicts, e.g. the merge operation adds only files which do not exist on master,
a common way to go is:

\begin{lstlisting}
	$\dollar$ git switch master             # make sure you are on the master 
																  # branch
	$\dollar$ git pull origin master        # update your master branch 
	$\dollar$ git merge <local_branch> 		 # create the merge commit
	$\dollar$ git push origin master        # publish your merge operation
\end{lstlisting}

In case you expect conflicts or you are not sure, it is advisable to separate the creation of 
a new commit form the merge process:
\begin{lstlisting}
	$\dollar$ git switch master    # make sure you are on the master branch 
	$\dollar$ git pull             # update your master branch
	$\dollar$ git merge --no-ff --no-commit <local_branch_name> # postpone commit creation
\end{lstlisting}

The option --no-ff means no fast forward and --no-commit is self explanatory.
Checking the Git status reveals if conflicts occurred during the merge process:
\begin{lstlisting}
	$\dollar$ git status
\end{lstlisting} 

You can call your favorite editor (I use VS Code, see \cref{chapter:Miscellaneous.1} for how to 
change your editor) to eventually solve the conflicts with
\begin{lstlisting}
	$\dollar$ git merge tool
\end{lstlisting} 
and afterwards commit everything:
\begin{lstlisting}
	$\dollar$ git commit -m"merge branch xxx into master"   # create commit  
	$\dollar$ git push    # publish your merge operation
\end{lstlisting}  

There is still one last issue a little bit unsatisfactory. Have a look at your local commit tree and compare 
it to the remote one as it is depicted in figure xxx after merging and publishing the merge commit:

hier das Bild mit den zwei verschiedenen commit Bäumen xxx

As we can clearly see, both trees differ a little bit. Especially for developers who only see the remote 
commit tree it is hard to understand what happens in the last commit. It would be nice if we could publish all
commits under our local development branch, too (the B commits). Therefore, we take all commits of our development branch
and put them at the end of the master branch. In other words, we create a new base commit for our local 
branch or we rebase all those commits, see figure xxx which illustrates this process. 

hier das rebasing bild rein

Putting it all together finally leads to the following command sequence:
\begin{lstlisting}
	$\dollar$ git checkout master
	$\dollar$ git pull
	$\dollar$ git checkout test
	$\dollar$ git pull
	$\dollar$ git rebase -i master
	$\dollar$ git checkout master
	$\dollar$ git merge --no-ff --no-commit <local_branch_name>                             
\end{lstlisting}
Again check the status with
\begin{lstlisting}
	$\dollar$ git status
\end{lstlisting}
and solve the conflicts. Then create the merge commit and publish everything:
\begin{lstlisting}
	$\dollar$ git commit -m"merge and rebade branch xxx into master"  
	$\dollar$ git push 
\end{lstlisting}  